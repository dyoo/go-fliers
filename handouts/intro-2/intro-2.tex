\documentclass{article}
\usepackage{graphicx}
\usepackage{url}

%% http://www.ctan.org/tex-archive/help/Catalogue/entries/igo.html
\usepackage{igo}
\gobansize{13}

\usepackage[empty,cm]{fullpage}
\begin{document}

%% The first page includes a 9x9 board.
\centerline{\Huge WPI Go Beginner's Night}
\centerline{\Huge 7pm Friday, February 27, 2009}
\centerline{\Huge @ Campus Center, 2nd floor}
\vfill

\begin{center}
\includegraphics{gb9.epsi}
\end{center}

\vfill

\newpage

\section*{Eye, eye, eye, eye, stayin' alive, stayin' alive...}

Go is often called the ``surrounding game'', and for good reason!  The
capturing rule applies when we surround stones.  But another thing we
surround on the board is empty space.  Groups that surround a large
amount of empty space resist capture.  In fact, if we surround in a
particular way, our groups become altogether immune.
\begin{center}
\shortstack{
\cleargoban
\white{a3,b3,b2,b1,c3,d3,e3,e2,e1}
\showgoban
}
\hspace{.2in}
\shortstack{
\cleargoban
\white{a2,a3,b3,b1,c3,c2,c1}
\showgoban}
\hspace{.2in}
%
\shortstack{
\cleargoban
\white{c6,c5,c4,d6,d4,e5,e7,f5,f6,f7}
\showgoban
}
%
\hspace{.2in}
\shortstack{
\cleargoban
\white{a10,a11,b10,b12,c10,c11,c12,c13}
\showgoban
}
%% diagrams here
\end{center}
The distinguishing feature about these invincible groups is that
they've surrounded two distinct empty areas; we call these areas
\emph{eyes}.  Black can't capture because suicidal moves are
prohibited.  If a group can form two eyes, even despite the opponent's
best resistance, then that group is \emph{alive}.


\section*{Puzzles}
Here are a few puzzles about the \emph{life and death} of groups.
\begin{center}
\shortstack{
\cleargoban
\white{a3,b3,c3,d3,e3,e2,e1}
\black{a2,b2,c2,d2,d1}
\showgoban\\
Black's move.\\Where can Black form eyes?
}
\hspace{.3in}
\shortstack{
\showgoban\\
Black's move.\\Where can Black form eyes?
}
\hspace{.3in}
\shortstack{
\showgoban\\
White's move.\\Can White interfere with Black's eyes?
}
\end{center}




\section*{Scoring}

Here is an example of a game at its end.
%% diagram here
Once the borders between Black and White are solid, how do we score a game?
\begin{enumerate}
\item Groups that can't avoid capture are given to the opponent.

\item Each player tallies the empty space they've completely surrounded
  by his or her groups.  This is \emph{territory}; territory plus the
  number of captures sums to the player's score.
\end{enumerate}




\section*{For more information...}
\begin{itemize}
\item \emph{WPI Go} (\url{http://go.hashcollision.org/}) is our homepage.

\item \emph{Sensei's Library} (\url{http://senseis.xmp.net/}), a wiki
  devoted to the game.

\item \emph{The Interactive Way to Go}
  (\url{http://playgo.to/interactive/}), which gives a more leisurely
  introduction.
\end{itemize}
\end{document}
