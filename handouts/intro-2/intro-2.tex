\documentclass{article}
\usepackage{graphicx}
\usepackage{url}

%% http://www.ctan.org/tex-archive/help/Catalogue/entries/igo.html
\usepackage{igo}
\gobansize{9}

\usepackage[empty,cm]{fullpage}
\begin{document}

%% The first page includes a 9x9 board.
\centerline{\Huge WPI Go Beginner's Night}
\centerline{\Huge 7pm Friday, February 27, 2009}
\centerline{\Huge @ Campus Center, 2nd floor}
\vfill

\begin{center}
\includegraphics{gb9.epsi}
\end{center}

\vfill

\newpage

\section*{Introduction}

Go is often called the ``surrounding game''.  Group capture happens
when we surround stones.  Another thing we can surround is vacant
space, and it turns out that surrounding this space can be
very useful.


\section*{Eye, eye, eye, eye, stayin' alive, stayin' alive...}

Concretely, there's a advantage for having a group surround a large
amount of empty space: such a group is naturally harder to capture
since it's adjacent to so much space.  In fact, if we surround in a
particular way, our groups become completely immune to capture.

%% diagrams here

The distinguishing feature about these invincible groups is this:
they've been able to surround at least two separate empty spaces.
Black can't play to completely fill \igotriangle, because of the
prohibition against playing suicidal moves.  If a group has two eyes,
it can dodge capture, even despite an opponent's best resistance.

%% put puzzles here on life and death

%% Black's move.  Can Black form two eyes?

%% Black's move.  Does Black need to play at ... immediately, or can
%% he wait?

%% White's move.  Does White need to play at ... immediately, or can
%% she wait?

%% Black's move.  Can Black prevent White from making two eyes?



\section*{Scoring}

Here is an example of a game at its end.
%% diagram here
The borders between Black and White are solid, and neither feels they
can make productive moves any more.  How do we score the game?
\begin{itemize}
\end{itemize}




\section*{For more information...}
\begin{itemize}
\item \emph{WPI Go} (\url{http://go.hashcollision.org/}) is our homepage.

\item \emph{Sensei's Library} (\url{http://senseis.xmp.net/}), a wiki
  devoted to the game.

\item \emph{The Interactive Way to Go}
  (\url{http://playgo.to/interactive/}), which gives a more leisurely
  introduction.
\end{itemize}
\end{document}
