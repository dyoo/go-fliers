\documentclass{article}
\usepackage{graphicx}
\usepackage{url}

%% http://www.ctan.org/tex-archive/help/Catalogue/entries/igo.html
\usepackage{igo}
\gobansize{13}

\usepackage[empty,cm]{fullpage}
\begin{document}

%% The first page includes a 9x9 board.
\centerline{\Huge WPI Go Beginner's Night}
\centerline{\Huge 7pm Friday, February 27, 2009}
\centerline{\Huge @ Campus Center, 2nd floor}
\vfill

\begin{center}
\includegraphics{gb9.epsi}
\end{center}

\vfill

\newpage

\section*{Eye, eye, eye, eye, stayin' alive, stayin' alive...}

Go is often named the ``surrounding game''.  The name is appropriate: we
surround stones to capture them.  But another thing we surround on the
board is empty space.  Groups that surround a large amount of empty
space resist capture.  In fact, if we surround in a particular way,
our groups become altogether immune.
\begin{center}
\shortstack{
\cleargoban
\white{a3,b3,b2,b1,c3,d3,e3,e2,e1}
\showgoban
}
\hspace{.2in}
\shortstack{
\cleargoban
\white{a2,a3,b3,b1,c3,c2,c1}
\showgoban}
\hspace{.2in}
%
\shortstack{
\cleargoban
\white{c6,c5,c4,d6,d4,e5,e7,f5,f6,f7}
\showgoban
}
%
\hspace{.2in}
\shortstack{
\cleargoban
\white{a10,a11,b10,b12,c10,c11,c12,c13}
\showgoban
}
%% diagrams here
\end{center}
These invincible groups have surrounded at least two distinct empty
areas.  Because suicidal moves are prohibited, Black has no
possibility to capture these groups.  We call these empty surrounded
areas \emph{eyes}.  If a group can form at least two eyes, in spite of
an opponent's best resistance, then that group is \emph{alive}.
Groups that have no chance to make two eyes are most likely
\emph{dead}.  (There is a special case of life that doesn't involve
two eyes, which we'll talk about in a future handout.)


Some groups waver on the line between life and death.  Let's take a
look at a few of these.
\begin{center}
\shortstack{
\cleargoban
\white{a3,b3,c3,d3,e3,e2,e1}
\black{a2,b2,c2,d2,d1}
\showgoban\\
Black's move.\\Where can Black form eyes?
}
\hspace{.3in}
\shortstack{
\cleargoban
\white{c12,c11,d11,e11,f11,g11,h11,j11,j12,j13}
\black{d12,e12,f12,f13,g12,h12,h13}
\showgoban\\
Black's move.\\Where can Black form eyes?
}
\hspace{.3in}
\shortstack{
\showgoban\\
White's move.\\Can White interfere with Black's eyes?\\
(\small{\it Your opponent's good move is your good move.})
}
\end{center}
These particular three examples are simple, but as you might imagine,
\emph{Life and Death} puzzles can get much more elaborate.



\section*{Scoring}

Once the borders between Black and White are solid, how do we score a game?
\begin{enumerate}
\item Groups that can't avoid capture are given to the opponent.  If
there's disagreement whether a group has died or not, keep playing
until mutual agreement is reached.

\item Each player tallies the empty space they've completely surrounded
  by his or her groups.  This is \emph{territory}; we add territory
  and the number of captures together to get the total score.
\end{enumerate}
%
Here is an example of a game that's reached its end.
\begin{center}
\shortstack{
\cleargoban
\gobansize{9}
\black[1]{d7,f4,c4,f7,f6,g6,f5,g5,g7,f8}
\black[11]{e7,g8,e4,f3,e3,e2,d2,f2,c3,c9}
\black[21]{c8,e9,b9,d9,b7,e8,e1,f1,d1,d8}
\cleargobansymbols
\black[\igotriangle]{g7}
\showfullgoban\\
Both players agree that \blackstone[\igotriangle] is dead.}
\hspace{.3in}
\shortstack{
\clear{g7}
\showfullgoban\\
\blackstone[\igotriangle] is given to white.}
\hspace{.3in}
\shortstack{
\gobansymbol{a1,a2,a3,a4,a5,a6,a7,a8,a9,b1,b2,b3,b4,b5,b6,b8,c1,c2,c5,c6,c7,d3,d4,d5,d6,e5,e6}{b}
\gobansymbol{f9,g1,g2,g3,g4,g7,g9,h1,h2,h3,h4,h5,h6,h7,h8,h9,j1,j2,j3,j4,j5,j6,j7,j8,j9}{w}
\showfullgoban\\
Black and White count territories.}
\end{center}
Black has 27 territories, and his score is 27.  White has 25
territories, and since White received 1 captured stone, her total
score is 26.

\section*{For more information...}
\begin{itemize}
\item \emph{WPI Go} (\url{http://go.hashcollision.org/}) is our homepage.

\item \emph{Sensei's Library} (\url{http://senseis.xmp.net/}), a wiki
  devoted to the game.

\item \emph{The Interactive Way to Go}
  (\url{http://playgo.to/interactive/}), which gives a more leisurely
  introduction.

\item \emph{Go Problems}
 (\url{http://www.goproblems.com}) is a database of puzzles, many
  which involve the life and death of groups.

\end{itemize}
\end{document}
