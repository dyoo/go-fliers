\documentclass{article}
\usepackage{graphicx}
\usepackage{url}

%% http://www.ctan.org/tex-archive/help/Catalogue/entries/igo.html
\usepackage{igo}
\gobansize{9}

\usepackage[empty,cm]{fullpage}
\begin{document}

%% The first page includes a 9x9 board.
\centerline{\Huge WPI Go Beginner's Night}
\centerline{\Huge 7pm Friday, February 27, 2009}
\centerline{\Huge @ Campus Center, 2nd floor}
\vfill

\begin{center}
\includegraphics{gb9.epsi}
\end{center}

\vfill

\newpage

\section*{Eye, eye, eye, eye, stayin' alive, stayin' alive...}

Go is often called the ``surrounding game'', and for good reason!  The
capturing rule applies when we surround stones.  But another thing we
surround is empty space.  We may have noticed that groups that
surround a large amount of empty space are harder to capture.

In fact, if we surround in a particular way, our groups
become altogether immune.
\begin{center}
\shortstack{
\cleargoban
\black{a2,b2,b1,c2,d2,e2,e1}
\showgoban
}
\shortstack{
\cleargoban
\showgoban
}
\shortstack{
\cleargoban
\showgoban
}
%% diagrams here
\end{center}
The distinguishing feature about these invincible groups is that
they've surrounded two distinct empty areas.  Black can't play to
completely fill either area because suicidal moves are prohibited.  If
a group can form two eyes, even despite the opponent's best
resistance, then that group is \emph{alive}.


\section*{Puzzles}
Here are a few puzzles about the \emph{life and death} of groups.  Try to
solve them:
\begin{center}

%% put puzzles here on life and death

%% 

%% 
%%
%% 

\shortstack{
\showgoban\\
Black's move.\\Where can Black form eyes?
}
\hspace{.3in}
\shortstack{
\showgoban\\
Black's move.\\Where can Black form eyes?
}
\hspace{.3in}
\shortstack{
\showgoban\\
White's move.\\Can White interfere with Black's eyes?
}
\end{center}




\section*{Scoring}

Here is an example of a game at its end.
%% diagram here
The borders between Black and White are solid, and neither feels they
can make productive moves any more.  How do we score the game?
\begin{itemize}
\item Groups whose capture is inevitable are removed and given to the
  opponent.

\item Each player counts the whitespace sounded by their remaining
  groups.  This \emph{territory}, plus the number of captures, forms
  the player's score.
\end{itemize}




\section*{For more information...}
\begin{itemize}
\item \emph{WPI Go} (\url{http://go.hashcollision.org/}) is our homepage.

\item \emph{Sensei's Library} (\url{http://senseis.xmp.net/}), a wiki
  devoted to the game.

\item \emph{The Interactive Way to Go}
  (\url{http://playgo.to/interactive/}), which gives a more leisurely
  introduction.
\end{itemize}
\end{document}
