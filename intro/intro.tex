\documentclass{article}
\usepackage{graphicx}
\usepackage{url}

%% http://www.ctan.org/tex-archive/help/Catalogue/entries/igo.html
\usepackage{igo}
\gobansize{9}

\usepackage[empty,cm]{fullpage}
\begin{document}

%% The first page includes a 9x9 board.
\hbox{}
\vfill

\begin{center}
\includegraphics{gb9.epsi}
\end{center}

\vfill

\newpage

%% The next page includes a small description of the capturing rules,
%% a few puzzles, and references for more information.
\section*{Introduction}
Go is a board game between black and white.  Like Chess, black and
white take turns playing on the intersections of the board.  Unlike
Chess, the board starts out empty. The goal is not necesarily to
capture the opponent's pieces, but to take majority control over the
board.  That being said, control of the board is intimately related to
capture, so let's show how to capture stones.

Adjacent stones of the same color form a \emph{group} that live or die
together as one.  (Diagonals don't count!)

\begin{center}
\cleargoban
\gobansize{19}
\shortstack{
\black{a1,a2,b3,c3,d3,d2}
\showgoban\\Two black groups.}
\hspace{.5in}%
\shortstack{
\cleargoban
\white{d6,d5,d4,e4,f4,f5,f6,d6,e6}
\showgoban\\One white group.}
\hspace{.5in}%
\shortstack{
\cleargoban
\black{d9,d10,e8,f9,g8,h9,h10}
\white{k10,l10,m10,n10,o10,l9,l8,n9,n8}
\showgoban\\Five black groups and one white group.}
\end{center}
%
A group stays on the board as long as it's adjacent to a vacant
intersection.  But if the opposing player places a stone that occupies
the last empty space of a group, that group is captured and removed
from the board.


\begin{center}
\cleargoban
\shortstack{
\black{}
\showgoban\\Two black groups.}
\hspace{1in}%
\shortstack{
\cleargoban
\black{}
\showfullgoban\\One white group.}
\hspace{1in}%
\shortstack{
\cleargoban
\black{}
\showfullgoban\\One white and one black group.}
\end{center}



\section*{Puzzles}
\begin{center}
\cleargoban
\shortstack{
\black{}
\showgoban\\Two black groups.}
\hspace{1in}%
\shortstack{
\cleargoban
\black{}
\showfullgoban\\One white group.}
\hspace{1in}%
\shortstack{
\cleargoban
\black{}
\showfullgoban\\One white and one black group.}
\end{center}




%% Game taken from http://www.societies.cam.ac.uk/cugos/go/rules_06.html
\section*{A 9x9 Game}
To help give a quick (but incomplete!) impression of the game, here's
an example of a 9x9 game.
%
\begin{center}
\shortstack{
\gobansize{9}
\black[1]{d7,f4,c4,f7,f6,g6,f5,g5,g7,f8}
\showfullgoban\\Moves 1--10.}
\hspace{1in}%
\shortstack{
\cleargobansymbols
\black[11]{e7,g8,e4,f3,e3,e2,d2,f2,c3,c9}
\showfullgoban\\Moves 11--20.}
\hspace{1in}%
\shortstack{
\cleargobansymbols
\black[21]{c8,e9,b9,d9,b7,e8,e1,f1,d1,d8}
\showfullgoban\\Moves 21--30.}
\end{center}
%
Black and white jostle and push against each other.  Black emphasizes
the left side of the board, and white the right side.  We'll talk
about how to evaluate the score in our next handout.  But as a teaser:
at the end of the game, Black gets 26 points, and White gets 25
points, so Black wins.


\section*{For more information...}
\begin{itemize}

\item \emph{Sensei's Library} (\url{http://senseis.xmp.net/}), a wiki
  devoted to the game.

\item \emph{The Interactive Way to Go}
  (\url{http://playgo.to/interactive/}), which gives a more leisurely
  introduction.
\end{itemize}
\end{document}
